\section{Guide}
\subsection{Notes}
\begin{itemize}
\item Don't forget to solve the problem in reverse (i.e deleting->adding or adding->deleting, ...etc)
\item Max flow is just choosing the maximum number of paths between source and sink
\item If you have a problem that tells you choose a[i] or b[i] (or a range) choose one of them initially and play a take or leave on the other
\item If the problem tells you to do something cyclic solving it for x + x
\item Problems that are close to NP problems sometimes have greedy solutions for large input i.e n >=20-30
\item in case of merging between sets try bitsets (i.e i + j or sth)
\item If you have a TLE soln using bitset might help
\item If everything else fails think Brute force or randomization
\end{itemize}
\hrulefill
\subsection{Assignment Problems}
\begin{itemize}
\item If you see a problem that tells you out of N choose K that has some property (think flows or aliens trick)
\item If you see a problem that tells for some X choose a Y (think flows)
\item If the problem tells you to choose a Y from L->R (think range flow i.e putting edges between the same layer)
\end{itemize}
\hrulefill
\subsection{XOR problems}
\begin{itemize}
\item If the problem tells your something about choosing an XOR of a subset (think FWHT or XOR-basis)
\item If the problem tells you about getting XOR of a tree path let a[i] = XOR tree from root to i and solve this as an array
\item If the problem tells you range XOR sth it's better to have prefix XOR and make it pairs XOR.
\end{itemize}
\hrulefill
\subsection{Decompositions}
\begin{itemize}
\item If a problem is a asking you to calculate the answer after K steps you can calculate the answer for K%squareRoot and for K/SquareRoot
\item If the nubmer of queries is significintly larger than updates or vice versa you can use square root Decompositions to give advantage to one over the other
\end{itemize}
\hrulefill
\subsection{Strings}
\begin{itemize}
\item Longest Common Substring is easier with suffix automaton 
\item Problems that tell you cound stuff that appears X times or count appearnces (Use suffixr links)
\item Problems that tell you find the largest substring with some property (Use Suffix links)
\item Remember suffix links are the same as aho corasic failure links (you can memoize them with dp)
\item Problems that ask you to get the k-th string (can be either suffix automaton or array)
\item Longest Common Prefix is mostly a (suffix automaton-array) thing
\item try thinking bitsets
\end{itemize}
\hrulefill 
\subsection{Trees}
\begin{itemize}
\item For problems that ask you to count stuff in a substree think (Euler Tour with RQ - Small to Large - DSU on Trees - PersistentSegTree)
\item Note that the farthest node to any node in the tree is one of the two diameter heads
\item In case of asking F(node, x) for each node it's probably DP on Trees
\end{itemize}
\hrulefill 
\subsection{Flows}
\begin{itemize}
\item If you want to make a K-covering instead of consdirign lit edges consider non-lit edges
\item To get mincost while mainting a flow network (note that flows are batched together according to cost)
\item If the problem asks you to choose some stuff the minimizes use Min Cut (If maximizes sum up stuff and subtract min cut)
\end{itemize}
\hrulefill 
\subsection{Geometry}
\begin{itemize}
\item Manhattan to King distance (x,y) -> (x+y, x-y)
\item Lattice points on line: gcd(dx,dy) + 1
\item Pick's theorem: $A = I + \frac{B}{2} - 1$
\item cosine rule: $C^2=A^2+B^2-2AB \times cos(c)$
\item Rotation around axis: $R = (cos(a) \times Id + sin(a) \times cross U + (1 - cos(a)) \times outer U)$
\item Triangulation of n-gon = Catalan (n-2)
\end{itemize}
\hrulefill
\subsection{Area}
\begin{itemize}
\item triangle = $\sqrt(S \times (S-A) \times (S-B) \times (S-C))$, S = PERIMETER/2
\item triangle = $r \times S$, r = radius of inscribed circle
\item ellipse = $\pi \times r_1 \times r_2$
\item sector = $\frac{(r^2 \times a)}{2}$
\item circular cap = $\frac{R^2 \times (a-sin(a))}{2}$
\item prsim = $perimeter(B) L + 2 area(B)$
\item sphere = $4 \pi r^2$
\end{itemize}
\hrulefill
\subsection{Volume}
\begin{itemize}
\item Right circular cylinder = $\pi r^2 h$ 
\item Pyramid = $\frac{Bh}{3}$
\item Right circular cone = $\frac{\pi r^2 h}{3}$
\item Sphere = $\frac{4}{3} \pi r^2 h$
\item Sphere sector= $\frac{2}{3} \pi r^2h = \frac{2}{3} \pi r^3 (1-cos(a))$
\item Sphere cap = $\frac{\pi h^2 (3r-h)}{3}$
\end{itemize}
\hrulefill
\subsection{Combinatorics}
\begin{itemize}
\item Cayley formula: number of forest with k trees where first k nodes belongs to different trees = $k n^{n-k-1}$ . Multinomial theorem for trees of given degree sequence ${\binom{n}{d_i}}$
\item Prufer sequence (M5da calls it parent array)
\item  K-Cyclic permutation = ${\binom{n}{k}} \times (k-1)!$
\item Stirling numbers $S(n,k) = k \times S(n-1,k) + S(n,k-1)$ number of way to partition n in k sets.
\item Bell number $B_n = \sum_1^n (n-1, k) B_k$ 
\item \# ways to make a graph with k components connected $n^{k-2} \times \prod_{i=1}^k s_i$
\item Arithmetic-geometric-progression $S_n = \frac{A_1 \times G_1 - A_{n+1} \times G_{n+1}}{1-r} + \frac{dr}{(1-r)^2} \times (G_1-G_{n+1})$
\end{itemize}
\hrulefill
\subsection{Graph Theory}
\begin{itemize}
\item Graph realization problem: sorted decreasing degrees: $\sum_1^k d_i = k(k-1)+sum_(k+1)^n \min(d_i,k)$ (first k form clique and all other nodes are connected to them).
\item Euler formula: $v+f = e+c+1$
\item \# perfect matching in bipartite graph, $DP[S][j] = DP[S][j-1] + DP[S/{v}][j-1]$ for all v connected to the j node.
\end{itemize}
\hrulefill
\subsection{Max flow with lower bound}
\begin{itemize}
\item feasible flow in a network with both upper and lower capacity constraints, no source or sink: capacities are changed to upper bound - lower bound. Add a new source and a sink. let M[v] = (sum of lower bounds of ingoing edges to v) - (sum of lower bounds of outgoing edges from v). For all v, if M[v]>0 then add edge (S,v) with capacity M, otherwise add (v,T) with capacity -M. If all outgoing edges from S are full, then a feasible flow exists, it is the flow plus the original lower bounds.
\item maximum flow in a network with both upper and lower capacity constraints, with source s and sink t: add edge (t,s) with capacity infinity. Binary search for the lower bound, check whether a feasible exists for a network WITHOUT source or sink (B).
\end{itemize}
\hrulefill
\subsection{Joseph problem}
${\displaystyle g(n,k)={\begin{cases}
 0 & {\text{if }}n=1\\
 (g(n-1,k)+k){\bmod {n}}&{\text{if }} 1<n<k\\
 \left\lfloor {\frac {k((g(n',k)-n{\bmod {k}}){\bmod {n}}')}{k-1}}\right\rfloor {\text{where }}n'=n-\left\lfloor {\frac {n}{k}}\right\rfloor &{\text{if }}k\leq n\\
 \end{cases}}}$\\
\hrulefill